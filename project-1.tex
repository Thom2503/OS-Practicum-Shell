\documentclass[a4paper]{article}

\usepackage{a4wide}

\usepackage{tikz}

\usepackage{color}

\usepackage{enumerate}

\usepackage{array}

\usepackage{ebproof}

\usepackage{listings}

\usepackage[american]{babel}

\title{Operating System Concepts\\Project Shell}

% Replace the placeholders by your real name and student number
% This is for the tutorial and distribution among the TAs
\author{Jan Mes \hfill \textcolor{red}{s1116707} \\
        Thom Veldhuis \textcolor{red}{s1173167}}

% In LaTeX everything before \begin{document} is called pre-amble.
% This is where you put all important settings. The real document
% starts after \begin{document}.
\begin{document}
\maketitle 
% \maketitle makes sure that the title is shown on the first page of
% the document.
\subsection*{1: implementation}
First we check if the command given is either \verb|exit| or \verb|cd| by looking at the first part of the first command in the expression.
We check if it is one of these commands, because they are trivial to implement and do not require much further logic. 
They should just be executed immediately. \smallbreak
For other possible commands, we loop over the \verb|expression.commands| vector. 
For every command in this vector, we create a pipe with the \verb|pipe()| system call if it is not the last command in the vector, 
and we create a child process with the \verb|fork()| system call. \smallbreak
Inside the child process we set the in- and output of the command, and execute it. 
We do this by first checking if the \verb|expression.inputFromFile| string is nonempty. 
If it is, and we are in the first loop (i.e. the first command), we open the file and set \verb|stdin| to it. \\
Then we check if the \verb|expression.outputToFile| string is nonempty.
If it is, and we are in the last loop, we open the file and set \verb|stdout| to it. \\
\verb|stdin| might need to be redirected to read from a previous pipe,
so we check if our \verb|prev_file_descriptor| was set (optionally we could also just check if we are on the first loop/command).
If it was, we redirect \verb|stdin| to the \verb|prev_file_descriptor|. \\
Finally we check if we are on the last command. 
If we are not, the \verb|stdout| is redirected to the input of the pipe, \verb|file_descriptor[1]|, 
so that the next command can use it as input for the next command by reading from the pipe. \\
Now we just need to execute the command, and then the child process is done. \smallbreak
While this is happening, the parent process closes the file descriptors, 
but crucially not \verb|file_descriptor[0]| unless on the final command, as this would prevent the next iteration from accessing it.
Finally, to implement background processes, we check wether the \verb|expression.background| boolean is false.
If it is the command should not be run as a background process and we call \verb|waitpid()| to wait for the child process to complete.
Otherwise it is a background process and we should not wait for the child processes.

\end{document}